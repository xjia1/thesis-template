\begin{abstract}
\noindent
Lepidopteron wing scales with their periodic dielectric 
structures producing glaring iridescent colors attract 
great attention in bio-photonic devices design and 
fabrication during the past decade. One inevitable 
drawback of these bio-templates is the lack in 
flexibility limited by the wing scales’ inborn inner 
microstructure. We design two novel routes to control 
and modify the original structure of Chrysiridia 
rhipheus (sunset moth) wing scales and thus their 
optical properties, using a kind of biocompatible 
environment sensitive interpenetrating polymer network 
(IPN) chitosan/PVA. The immobilized wing scales' 
visible reflectance is responsive to both the electric 
field and the pH condition, owing to their inner 
microstructure change induced by the IPN volume change 
in the swell/deswell process driven by electric field 
and pH condition. Using electric field as the driving 
force, we obtain a total \textasciitilde 150nm visible 
reflectance shift within several minutes, which can be 
used as an optical switch or electric field sensor; in 
pH driving system, we obtain a total \textasciitilde
260nm visible reflectance shift, and prominent 
sensitivity of moth wing scale indicator's optical 
property in week basic condition (pH=8-10) promises an 
in vivo bio-sensing pH monitor that can be used in 
general bio-medical and bio-controlled applications. 
Our multi-responsive optical sensors for E-field and 
pH condition broaden the natural species' pool for 
functional structure selection, and provide designable 
and controllable bio-inspired material solutions 
according to specific practical demands.
\end{abstract}

\keywords{Biomedical Sensor, Lepidoptera scales, Nature photonics, Optical sensor/indicator, Eletric field sensitive; pH condition sensitive; Interpenetrating polymer network}
